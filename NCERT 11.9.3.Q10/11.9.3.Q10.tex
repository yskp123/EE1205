% \iffalse
\let\negmedspace\undefined
\let\negthickspace\undefined
\documentclass[journal,12pt,twocolumn]{IEEEtran}
\usepackage{cite}
\usepackage{amsmath,amssymb,amsfonts,amsthm}
\usepackage{algorithmic}
\usepackage{graphicx}
\usepackage{textcomp}
\usepackage{xcolor}
\usepackage{txfonts}
\usepackage{listings}
\usepackage{enumitem}
\usepackage{mathtools}
\usepackage{gensymb}
\usepackage{comment}
\usepackage[breaklinks=true]{hyperref}
\usepackage{tkz-euclide} 
\usepackage{listings}
\usepackage{gvv}                                        
\def\inputGnumericTable{}                                 
\usepackage[latin1]{inputenc}                                
\usepackage{color}                                            
\usepackage{array}                                            
\usepackage{longtable}                                       
\usepackage{calc}                                             
\usepackage{multirow}                                         
\usepackage{hhline}                                           
\usepackage{ifthen}                                           
\usepackage{lscape}

\newtheorem{theorem}{Theorem}[section]
\newtheorem{problem}{Problem}
\newtheorem{proposition}{Proposition}[section]
\newtheorem{lemma}{Lemma}[section]
\newtheorem{corollary}[theorem]{Corollary}
\newtheorem{example}{Example}[section]
\newtheorem{definition}[problem]{Definition}
\newcommand{\BEQA}{\begin{eqnarray}}
\newcommand{\EEQA}{\end{eqnarray}}
\newcommand{\define}{\stackrel{\triangle}{=}}
\theoremstyle{remark}
\newtheorem{rem}{Remark}
\begin{document}

\bibliographystyle{IEEEtran}
\vspace{3cm}

\title{NCERT 11.9.3.Q10}
\author{EE23BTECH11224 - Sri Krishna Prabhas Yadla$^{*}$% <-this % stops a space
}
\maketitle
\newpage
\bigskip

\renewcommand{\thefigure}{\theenumi}
\renewcommand{\thetable}{\theenumi}


\vspace{3cm}
\textbf{Question:} Find the sum to indicated number of terms in the geometric progression \(x^3,x^5,x^7,...n\) terms (if \(x\neq\pm1\)).
\\
\solution
% \fi
Let $S(n)$ be the sum of the first n terms in G.P starting from $x(0)$. We have
\begin{align}
    x(n) &= x(0) \cdot r^n \\
    S(n) &= \sum_{k=0}^{n-1}x(k) \\
    &= x(0) \frac{r^n-1}{r-1} \text{    (for r$\neq$1)}
\end{align}
\begin{table}[h]
    \input{tables/inputs_table}
    \label{tab:table1}
    \caption{Given Inputs}
\end{table}
\newline
Hence the common ratio, $r$, can be calculated by
\begin{align}
    r = \frac{x(1)}{x(0)} = \frac{x^5}{x^3} = x^2 
\end{align}
Since $x\neq\pm1$, $r\neq1$,
\begin{align}
    S(n) &= x(0) \frac{r^n-1}{r-1} \\
   \therefore S(n) &= x^3 \frac{x^{2n}-1}{x^2-1}
\end{align}
\newpage
\begin{align*}
	x(n) \stackrel{\mathcal{Z}}{\longleftrightarrow} X(z)
\end{align*}

\begin{align}
	X(z) &= \sum_{n=-\infty}^{\infty}x(n)z^{-n} \\
	&= \sum_{n=0}^{\infty}x(n)z^{-n} \\
	&= \sum_{n=0}^{\infty}x(0)r^nz^{-n} \\
	&= \frac{x(0)}{1-rz^{-1}} \\
	&= \frac{x^3}{1-x^2z^{-1}}
\end{align}
The z transform is defined only when $\lvert x^2z^{-1} \rvert<1$. So, ROC : $|z|>x^2$.
\end{document}
