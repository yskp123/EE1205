% \iffalse
\let\negmedspace\undefined
\let\negthickspace\undefined
\documentclass[journal,12pt,twocolumn]{IEEEtran}
\usepackage{cite}
\usepackage{amsmath,amssymb,amsfonts,amsthm}
\usepackage{algorithmic}
\usepackage{graphicx}
\usepackage{textcomp}
\usepackage{xcolor}
\usepackage{txfonts}
\usepackage{listings}
\usepackage{enumitem}
\usepackage{mathtools}
\usepackage{gensymb}
\usepackage{comment}
\usepackage[breaklinks=true]{hyperref}
\usepackage{tkz-euclide} 
\usepackage{listings}
\usepackage{gvv}                                        
\def\inputGnumericTable{}                                 
\usepackage[latin1]{inputenc}                                
\usepackage{color}                                            
\usepackage{array}                                            
\usepackage{longtable}                                       
\usepackage{calc}                                             
\usepackage{multirow}                                         
\usepackage{hhline}                                           
\usepackage{ifthen}                                           
\usepackage{lscape}

\newtheorem{theorem}{Theorem}[section]
\newtheorem{problem}{Problem}
\newtheorem{proposition}{Proposition}[section]
\newtheorem{lemma}{Lemma}[section]
\newtheorem{corollary}[theorem]{Corollary}
\newtheorem{example}{Example}[section]
\newtheorem{definition}[problem]{Definition}
\newcommand{\BEQA}{\begin{eqnarray}}
\newcommand{\EEQA}{\end{eqnarray}}
\newcommand{\define}{\stackrel{\triangle}{=}}
\theoremstyle{remark}
\newtheorem{rem}{Remark}
\begin{document}

\bibliographystyle{IEEEtran}
\vspace{3cm}

\title{NCERT 12.10.Q21}
\author{EE23BTECH11224 - Sri Krishna Prabhas Yadla$^{*}$% <-this % stops a space
}
\maketitle
\newpage
\bigskip

\renewcommand{\thefigure}{\arabic{figure}}
\renewcommand{\thetable}{\arabic{table}}


\vspace{3cm}
\textbf{Question:}In deriving the single slit diffraction pattern, it was stated that the intensity is zero at angles of $n\lambda/a$. Justify this by suitably dividing the slit to bring out the cancellation.
\\
\solution
% \fi
\begin{table}[htbp]
\centering
\def\arraystrech{1.5}
\begin{tabular}{|p{2.5cm}|p{5.5cm}|}
\hline
\textbf{Parameters} & \textbf{Description} \\
\hline
$\lambda$ & Wavelength \\
\hline
$a$ & Slit width \\
\hline
$k$ & Wave number\\
\hline
$\omega$ & Angular frequency\\
\hline
$ds$ & Infinitesimally small part of slit at a distance s from top of the slit\\
\hline
$\Delta x$ & Path difference\\
\hline
$\theta$ & Angle of elevation of point P on screeen from slit\\
\hline
$dE$& Electric field at point P associated with light wavelet at $ds$\\
\hline
$c_1, c_2$ & Proportionality constants \\
\hline
$E_1$ & Amplitude of E_P\\
\hline
$\phi' $ & Phase difference between $E_P$ and incident light's electric field\\
\hline
$I$& Intensity at point P\\
\hline
$I_0$ & Intensity at central bright band\\
\hline
\end{tabular}

\caption{Variables used}
\label{tab:1.12_10_Q21}
\end{table}
\begin{align}
    k &= \frac{2\pi}{\lambda} \\
\Delta x &= s\cdot \sin{\theta}\\
    dE &= (c_1\cdot ds)\cos{(k(x+\Delta x)-\omega t)} \\
    &= c_1 \cos{(ks\sin{\theta} + kx-\omega t)} ds \\
    E_P &= \int_{0}^{a} c_1\cos{(ks\sin{\theta} + kx-\omega t)} ds \\
    &= \frac{c_1 \sin{(ks\sin{\theta} + kx-\omega t)}}{k\sin{\theta}}\Bigg|_{0}^{a}   \\
    &= \frac{c_1 \sin{(ka\sin{\theta} + kx-\omega t)}-c_1\sin{(kx-\omega t)}}{k\sin{\theta}} \\
    &= \frac{2c_1}{k\sin{\theta}}\sin{ \brak{\frac{ka\sin{\theta}}{2}}}\cos{\brak{kx-\omega t+ \frac{ka\sin{\theta}}{2}}} \\
    &= E_1 \cos{(kx-\omega t+ \phi')} \\
    \implies E_1 &= \frac{2c_1}{k\sin{\theta}}\sin{ \brak{\frac{ka\sin{\theta}}{2}}}
\end{align}
\begin{align}
    I &\propto E_1^2  \\
    I &= c_2 \cdot \brak{\frac{2c_1}{k\sin{\theta}}\sin{ \brak{ \frac{ka\sin{\theta}}{2}}}}^2 \\
    &= c_2c_1^2a^2 \frac{\sin^2{\brak{\frac{ka\sin{\theta}}{2}}}}{\brak{\frac{ka\sin{\theta}}{2}}^2}\\
    I_0 &= \lim_{\theta\to 0} I = c_2c_1^2a^2   \\
    I &= I_0 \frac{\sin^2{\beta}}{\beta^2}  \\
\implies    \beta &= \frac{ka\sin{\theta}}{2} = \frac{\pi a\sin{\theta}}{\lambda}
\end{align}
For zero intensity,
\begin{align}
    \beta &= n\pi \\
    \frac{\pi a\sin{\theta}}{\lambda} &= n\pi \\
    a\sin{\theta} &= n\lambda \\
    \sin{\theta} &\approx \theta \quad \text{(for small angles)} \\
    \theta &= \frac{n\lambda}{a}
\end{align}

\end{document}
